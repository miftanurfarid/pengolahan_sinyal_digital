\documentclass[pdflatex,compress,mathserif]{beamer}

%\usetheme[dark,framenumber,totalframenumber]{ElektroITK}
\usetheme[darktitle,framenumber,totalframenumber]{ElektroITK}

\usepackage[utf8]{inputenc}
\usepackage[T1]{fontenc}
\usepackage{lmodern}
\usepackage[bahasai]{babel}
\usepackage{amsmath}
\usepackage{amsfonts}
\usepackage{amssymb}
\usepackage{graphicx}
\usepackage{multicol}
\usepackage{extarrows}

\newcommand*{\Scale}[2][4]{\scalebox{#1}{$#2$}}%

\title{DIGITAL SIGNAL PROCESSING}
\subtitle{The Discrete Fourier Series}

\author{Mifta Nur Farid}

\begin{document}

\maketitle

\section{Introduction}

\begin{frame}
	\frametitle{Introduction}
	\begin{itemize}
		\item In Signal and Systems, we've been discussing the Fourier and Z-transforms.
		\item the Fourier and Z-transforms provide us with a set of important 		analytical tools for representing discrete time signals, and also for dealing with discrete time systems.
		\item For example, that through the use of the Fourier transform or the Z-transform, we could convert convolution in the time domain to multiplication in either the frequency domain in the Fourier transform case, or more generally, in the Z domain in the Z-transform case.
	\end{itemize}
\end{frame}

\begin{frame}{Introduction}
	\begin{itemize}
		\item It would be hard to implementing, for example, a discrete time system by first computing the Fourier transform of the sequences, multiplying the Fourier transforms together, and then computing the inverse transform.
		\item One of the reasons that that's obviously a difficult thing to do computationally is that we saw, as we discussed, for example, the Fourier transform, that the Fourier transform is a function of a continuous variable.
		\item The $ \Omega $ in the Fourier transform is a continuous variable. So
		that, in fact, if we wanted to compute the Fourier transform explicitly, we would have to compute it at an infinite number of frequencies. Similarly, we have a situation like that for the Z-transform.
	\end{itemize}
\end{frame}

\end{document}
