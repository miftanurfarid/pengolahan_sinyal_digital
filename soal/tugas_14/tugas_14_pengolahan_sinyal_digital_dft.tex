\documentclass[12pt,a4paper]{article}
\usepackage[utf8]{inputenc}
\usepackage[T1]{fontenc}
\usepackage{amsmath}
\usepackage{amsfonts}
\usepackage{amssymb}
\usepackage{graphicx}
\usepackage[indonesian]{babel}
\usepackage[left=2.00cm, right=2.00cm, top=2.00cm, bottom=2.00cm]{geometry}
\usepackage{float} 

\title{Tugas 14 - Pengolahan Sinyal Digital\\
	Disain Filter Digital IIR}

% remove spacing around date:
\usepackage{titling}
\predate{}
\postdate{}
\date{}

\begin{document}
	\maketitle
	\date{}
	\begin{enumerate}
		\item Diketahui suatu filter analog yang mana input $ x_a(t) $ dan output $ y_a(t) $ saling berkaitan dalam persamaan linear constant coefficient differential berikut:
		
		\[ \frac{dy_a(t)}{dt} + 0.9 y_a(t) = x_a(t) \]
		
		Filter digital didapatkan dengan cara mengganti turunan pertama dengan forward diference pertama sehingga
		
		\[ \left[ \frac{y(n+1) - y(n)}{T} \right] + 0.9 y(n) = x(n)  \]
		
		Asumsikan filter digital ini adalah kausal
		\begin{enumerate}
			\item Tentukan dan gambarkan magnitude dari frequency response analog filternya.
			\item Tentukan dan gambarkan magnitude dari frequency response digital filter untuk $ T = 10/9 $.
			\item Tentukan rentang nilai T dimana pada rentang tersebut filter digitalnya tidak stabil.
		\end{enumerate}
		\item Diketahui system function $ H_a (s) $ dari filter analog adalah
		\[ H_a (s) = \frac{s}{(s+1)(s+2)} \]
		Tentukan sistem function $ H(z) $ dari filter digital yang diperoleh dari filter analog dengan impulse invariance.
	\end{enumerate}
\end{document}