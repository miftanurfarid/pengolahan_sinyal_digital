\documentclass[pdflatex,compress]{beamer}

%\usetheme[dark,framenumber,totalframenumber]{ElektroITK}
\usetheme[darktitle,framenumber,totalframenumber]{ElektroITK}

\usepackage{lipsum}

\title{PENGOLAHAN SINYAL DIGITAL}
\subtitle{Pengantar Pengolahan Sinyal Digital}

\author{Tim Dosen Pengampu}

\begin{document}

\maketitle

\section{Kontrak Perkuliahan}

\subsection{Deskripsi Mata Kuliah}

\begin{frame}
	\frametitle{Deskripsi Mata Kuliah}
	\begin{itemize}
		\item Teknologi pengolahan sinyal digital (digital signal processing/ DSP) dan berbagai pengembangannya memberikan dampak terhadap kehidupan modern manusia. 
		\item Tanpa DSP, kita tidak akan memiliki audio atau video digital; digital recording; CD, DVD, MP3 player, iPhone, and iPad; kamera digital; telepon digital atau pun seluler; satelit digital dan TV; atau pun jaringan kabel dan nirkabel/ wireless.
		\item Peralatan medis menjadi lebih efisien. Tidak mungkin kita memperoleh hasil diagnosis yang presisi tanpa elektrokardiografi digital (ECG), atau radiografi digital dan segala citra medis.
	\end{itemize}
\end{frame}

\begin{frame}
	\begin{itemize}
		\item Kita juga hidup dengan cara yang berbeda sejak adanya sistem voice recognition, speech synthesis dan sistem editing gambar dan video.
		\item Tanpa DSP, ilmuan, engineer, dan teknokrat tidak akan memiliki tools yang powerfull untuk menganalisa dan memvisualisasikan data dan mendemonstrasikan desain mereka.
		\item Oleh sebab itu, pada Mata Kuliah ini mahasiswa akan dibekali konsep dasar dari pengolahan sinyal digital.
		\item Kemudian perancangan filter FIR dan IIR secara simulasi akan diajarkan. Dan di akhir perkuliahan akan diajarkan bagaimana menganalisa spektrum frekuensi hasil dari filter.
		\item Dengan mengikuti perkuliahan ini, diharapkan mahasiswa mampu merancang suatu sistem pengolahan digital pada segala bidang Teknik Elektro.
	\end{itemize}
\end{frame}

\subsection{Capaian Pembelajaran Mata Kuliah (CPMK)}

\begin{frame}
	\frametitle{Capaian Pembelajaran Mata Kuliah}
	\begin{itemize}
		\item Mahasiswa mampu merancang suatu sistem pengolahan digital secara simulasi.
	\end{itemize}
\end{frame}

\subsection{Bahan Kajian}

\begin{frame}{Bahan Kajian}
	\begin{enumerate}
		\item Sinyal dan sistem waktu diskrit.
		\item Transformasi Fourier waktu diskrit,
		\item Transformasi Z \& Invers Transformasi Z.
		\item Deret Fourier diskrit dan Transformasi Fourier Diskrit.
		\item Circular convolution.
		\item Representasi jaringan digital linear.
		\item Struktur jaringan dari sistem infinite impulse response (IIR).
	\end{enumerate}
\end{frame}

\begin{frame}{Bahan Kajian}
	\begin{enumerate}
		\setcounter{enumi}{7}
		\item Struktur jaringan dari sistem finite impulse response (FIR) dan efek parameter kuantisasi dalam struktur filter digital.
		\item Disain filter IIR digital.
		\item Filter Butterworth digital.
		\item Disain filter FIR digital.
		\item Komputasi dari transformasi Fourier diskrit/ DFT.
	\end{enumerate}
\end{frame}

\subsection{Pustaka}

\begin{frame}
	\frametitle{Pustaka}
	\begin{itemize}
		\item Pustakan utama:
		\begin{enumerate}
			\item Oppenheim, A. V., \& Schafer, R. W. (2014). Discrete-Time Signal Processing 3rd Edition. Boston: Pearson.
			\item Tan, L. \& Jiang, J. (2019). Digital Signal Processing. Fundamentals and Applications 3rd Edition. Cambridge: AcademiC Press.
		\end{enumerate}
		\item Pustaka pendukung
		\begin{enumerate}
			\item Schilling, R. J. \& Harris, S.L. (2011). Fundamentals of Digital Signal Processing using MATLAB. Boston:
			Cengage Learning.
			\item Referensi lainnya yang mendukung perkuliahan ini. Bisa berupa buku, jurnal, dll.
		\end{enumerate}
	\end{itemize}
\end{frame}

\subsection{Jenis dan Bobot Evaluasi}

\begin{frame}{Jenis dan Bobot Evaluasi}
	\begin{enumerate}
		\item Kehadiran: 10 \%
		\item Tugas: 10 \%
		\item Kuis: 20 \%
		\item UTS: 30 \%
		\item UAS: 30 \%
	\end{enumerate}
\end{frame}

\section{Pengantar Pengolahan Sinyal Digital}

\subsection{Pendahuluan}

\begin{frame}
	\frametitle{Pendahuluan}
	\begin{itemize}
		\item Pengolahan sinyal digital: salah satu lingkup dalam science and engineering yang berkembang pesat dalam kurung waktu 40 tahun.
		\item Merupakan hasil dari perkembangan teknologi komputer digital dan fabrikasi IC.
	\end{itemize}
\end{frame}

\end{document}
