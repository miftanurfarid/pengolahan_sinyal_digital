\documentclass[pdflatex,compress,mathserif]{beamer}

%\usetheme[dark,framenumber,totalframenumber]{ElektroITK}
\usetheme[darktitle,framenumber,totalframenumber]{ElektroITK}

\usepackage[utf8]{inputenc}
\usepackage[T1]{fontenc}
\usepackage{lmodern}
\usepackage[bahasai]{babel}
\usepackage{amsmath}
\usepackage{amsfonts}
\usepackage{amssymb}
\usepackage{graphicx}
\usepackage{multicol}
\usepackage{lipsum}

\newcommand*{\Scale}[2][4]{\scalebox{#1}{$#2$}}%

\title{Pengolahan Sinyal Digital}
\subtitle{Realisasi Filter Digital}

\author{Mifta Nur Farid}

\begin{document}

\maketitle

\section{Pengantar}

\begin{frame}
	\frametitle{Pengantar}
	\begin{itemize}
		\item Kita telah mempelajari tentang dasar-dasar analisis sinyal, serta karakterisasi dan analisis sistem waktu diskrit.
		\item Selanjutnya kita akan membahas lebih detail tentang desain dari sistem waktu diskrit.
		\item Desain filter digital mencakup seluruh kegiatan mulai dari menentukan filter digital yang dibutuhkan hingga membangun purwarupa, mengujinya, dan menggunakannya.
	\end{itemize}
\end{frame}

\begin{frame}{Pengantar}
	\begin{itemize}
		\item Tahapan-tahapan yang dilalui dalam mendesain filter digital antara lain:
		\begin{enumerate}
			\item Aproksimasi
			\item Realisasi
			\item Mempelajari kesalahan aritmatik
			\item Implementasi
		\end{enumerate}
	\end{itemize}
\end{frame}

\begin{frame}{Pengantar}
	\begin{itemize}
		\item Aproksimasi: proses membuat fungsi transfer yang memenuhi spesifikasi yang diinginkan, bisa dari respon amplitudo atau fasa, atau juga respon time-domain dari filternya.
		\item Metode yang ada untuk solusi aproksimasi dapat diklasifikasikan menjadi 2:
		\begin{enumerate}
			\item \textit{direct}: permasalahan diselesaikan langsung dalam domain-\textit{z}
			\item \textit{indirect}: didapatkan terlebih dahulu fungsi transfer waktu kontinyu-nya, kemudian dikonversi menjadi fungsi transfer waktu diskrit.
		\end{enumerate}
	\end{itemize}
\end{frame}

\begin{frame}{Pengantar}
	\begin{itemize}
		\item \textbf{Aproksimasi}: proses membuat fungsi transfer yang memenuhi spesifikasi yang diinginkan, bisa dari respon amplitudo atau fasa, atau juga respon time-domain dari filternya.
		\item Metode yang ada untuk solusi aproksimasi dapat diklasifikasikan menjadi 2:
		\begin{enumerate}
			\item \textit{direct}: permasalahan diselesaikan langsung dalam domain-\textit{z}
			\item \textit{indirect}: didapatkan terlebih dahulu fungsi transfer waktu kontinyu-nya, kemudian dikonversi menjadi fungsi transfer waktu diskrit.
		\end{enumerate}
		\item Filter non-rekursif selalu didesain melalui metode \textit{direct}.
		\item Sedangkan filter rekursif dapat didesain melalui metode \textit{direct} maupun \textit{indirect}
	\end{itemize}
\end{frame}

\begin{frame}{Pengantar}
	\begin{itemize}
		\item Metode aproksimasi juga dapat diklasifikasikan sebagai
		\begin{enumerate}
			\item \textit{closed-form}: permasalahan diselesaikan melalui langkah-langkah desain yang jumlahnya sedikit dengan menggunakan serangkaian persamaan \textit{closed-form}
			\item \textit{iterative}: solusi awal diasumsikan terlebih dahulu, kemudian dengan penggunaan metode optimisasi, solusi ditingkatkan hingga didapatkan kriteria desain yang sesuai.
		\end{enumerate}
		\item Secara umum, seorang desainer menggunakan metode aproksimasi yang
		\begin{enumerate}
			\item sederhana
			\item reliable/ handal
			\item menghasilkan desain yang presisi
			\item membutuhkan komputasi yang minimal, dll
		\end{enumerate}
	\end{itemize}
\end{frame}

\begin{frame}{Pengantar}
	\begin{itemize}
		\item \textbf{Realisasi} atau \textbf{sintesis}: proses menghasilkan struktur filter digital dari fungsi transfer atau karakterisasi lainnya dari suatu filter.
		\item Struktur filter = realisasi dari fungsi transfer.
		\item Sama seperti aproksimasi, metode realisasi dapat diklasifikasikan menjadi 2
		\begin{enumerate}
			\item \textit{direct}: realisasi didapatkan secara langsung dari fungsi transfer waktu diskrit.
			\item \textit{indirect}: realisasi didapatkan secara tidak langsung dari prototipe filter analog yang ekivalen.
		\end{enumerate}
	\end{itemize}
\end{frame}

\begin{frame}{Pengantar}
	\begin{itemize}
		\item Desainer biasanya tertarik pada realisasi yang
		\begin{enumerate}
			\item mudah diimplementasikan dalam very-large-scale integrated (VLSI) circuit form
			\item membutuhkan jumlah unit delay, adder, dan multiplier yang sedikit
			\item tidak terlalu terpengaruhi oleh finite-precision aritmatik di dalam pengimplementasiannya, dll
		\end{enumerate}
	\end{itemize}
\end{frame}

\begin{frame}{Pengantar}
	\begin{itemize}
		\item Setiap alat-alat yang menggunakan DSP di dalamnya, memiliki ketidaksempurnaan.
		\item Disebabkan oleh ketidakakuratan dari modelnya, toleransi komponen-komponennya, efek nonlinear yang tidak biasa dan tidak terduga, dll.
		\item Desain yang diterima adalah desain yang tidak sempurna namun tidak melanggar spesifikasi yang diinginkan.
		\item Ketidaksempurnaan dari filter digital umumnya disebabkan oleh ketidakpresisian numerik.
	\end{itemize}
\end{frame}

\begin{frame}{Pengantar}
	\begin{itemize}
		\item Saat proses aproksimasi, koefisian dari fungsi transfer ditentukan memiliki derajat presisi yang tinggi.
	\end{itemize}
\end{frame}

\end{document}
