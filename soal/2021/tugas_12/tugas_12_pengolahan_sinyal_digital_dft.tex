\documentclass[12pt,a4paper]{article}
\usepackage[utf8]{inputenc}
\usepackage[T1]{fontenc}
\usepackage{amsmath}
\usepackage{amsfonts}
\usepackage{amssymb}
\usepackage{graphicx}
\usepackage[indonesian]{babel}
\usepackage[left=2.00cm, right=2.00cm, top=2.00cm, bottom=2.00cm]{geometry}
\usepackage{float} 

\title{Tugas 12 - Pengolahan Sinyal Digital\\
	Struktur Jaringan untuk Sistem Infinite Impulse Response (IIR)}

% remove spacing around date:
\usepackage{titling}
\predate{}
\postdate{}
\date{}

\begin{document}
	\maketitle
	\date{}
	\begin{enumerate}
		\item Diketahui sistem linear kausal waktu diskrit yang direpresentasikan dalam persamaan beda berikut ini:
		\[ y(n) - \frac{3}{4}y(n-1) + \frac{1}{8}y(n-2) = x(n) + \frac{1}{3}x(n-1) \]
		Gambarlah grafik aliran sinyalnya untuk mengimplementasikan sistem ini dalam bentuk-bentuk berikut ini:
		\begin{enumerate}
			\item Direct Form I
			\item Direct Form II
			\item Cascade
			\item Parallel
		\end{enumerate}
	\end{enumerate}
\end{document}