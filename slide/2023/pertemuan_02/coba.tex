\documentclass{article}

\usepackage[indonesian]{babel}
\usepackage[utf8]{inputenc}
\usepackage{amsmath}
\usepackage[]{geometry}
\geometry{
    a4paper,
    total={170mm,257mm},
    left=20mm,
    top=20mm,
    }

\begin{document}

\title{Latihan Soal}

\author{Mifta Nur Farid}

\maketitle

\begin{enumerate}
    \item Tentukan DFT dari $\tilde{x}(nT)$ jika
    \begin{equation*}
        \tilde{x}(nT) = 
    \end{equation*}
\end{enumerate}

\begin{equation}
    \sum_{n=2}^6 W^{-kn} = W^{-2k} + W^{-3k} + W^{-4k} + W^{-5k} + W^{-6k}
\end{equation}

\begin{equation}
    \sum_{n=2}^6 W^{-kn} = W^{-2k} + (W^{-2k} \cdot W^{-k}) + (W^{-2k} \cdot W^{-2k}) + (W^{-2k} \cdot W^{-3k}) + (W^{-2k} \cdot W^{-4k})
\end{equation}

\begin{equation}
    \sum_{n=2}^6 W^{-kn} = W^{-2k} \cdot (1 + W^{-k} + W^{-2k} + W^{-3k} + W^{-4k})
\end{equation}

\begin{equation}
    \sum_{n=2}^6 W^{-kn} = W^{-2k} \cdot \frac{1 - (W^{-k})^5}{1 - W^{-k}}
\end{equation}

\begin{equation}
    \sum_{n=2}^6 W^{-kn} = W^{-2k} \cdot \frac{1 - W^{-5k}}{1 - W^{-k}}
\end{equation}

\end{document}